% دو بار پردازش با زی‌لاتک
\PassOptionsToPackage{pdfpagemode=FullScreen,hyperfootnotes=false}{hyperref}
\documentclass[10pt,xcolor=dvipsnames,professionalfont]{beamer}

\usepackage{amsmath,amssymb,amsfonts} 
\usepackage{tikz} 
\usepackage{graphicx}
\usepackage{listings}
\usepackage{ptext}

\usetheme{Warsaw}
\usefonttheme{serif}
\usecolortheme[named=blue]{structure}
\setbeamercovered{transparent}


%% برای قرار گرفتن شماره اسلاید
\expandafter\def\expandafter\insertshorttitle\expandafter{%
\insertshorttitle\hfill%
\inserttotalframenumber\,/\,\insertframenumber}
 %%%%%%%%%%%%%%
 %توجه: بسته هایی که نیاز هست قبل از بسته زی‌پرشین نوشته شود.
\usepackage{xepersian}
\settextfont{XB Zar}
%\setlatintextfont{Times New Roman}
\setdigitfont{Yas}

\defpersianfont\nas[Scale=1.5]{IranNastaliq}
\defpersianfont\xb[Scale=1.3]{XB Zar}

%\deflatinfont\tnr[Scale=1.2]{Times New Roman}
%\linespread{1.2} 

%%%%%%%%%%%%%%%
\definecolor{mygreen}{RGB}{28,172,0} 
\definecolor{mylilas}{RGB}{170,55,241}
\lstset{language=Matlab,
    breaklines=true,basicstyle=\ttfamily\scriptsize,
    morekeywords={matlab2tikz},
    keywordstyle=\color{blue},
    morekeywords=[2]{1}, keywordstyle=[2]{\color{black}},
    identifierstyle=\color{black},
    stringstyle=\color{mylilas},
    commentstyle=\color{mygreen},
    showstringspaces=false
}

% دستورات مورد نیاز برای استفاده از کلاس بیمر در command نوشته شده
\input{command}
\raggedleft
%%%%%%%%%%%%%%%%%%
\newcommand*{\co}[1]{\nas\textcolor{blue}{#1}}

%%%%
%%%%
\title{یادگیری عمیق با کراس}
\subtitle{مقدمه‌ای بر کراس}
\author[علیرضا بیکی]{\co{علیرضا بیکی}}

\institute{دانشگاه شیراز}
\date{\today}

%برای شماره خوردن قضیه،...
\setbeamertemplate{theorems}[numbered]
%بولد
\providetranslation{Theorem}{\large \bf قضیه}
\providetranslation{Definition}{تعریف}
\providetranslation{Example}{مثال}


\begin{document}

\begin{frame}
\maketitle
\end{frame}

\begin{frame}{فهرست مطالب}
\tableofcontents
\end{frame}


\section{مقدمه‌ای بر کراس}

\begin{frame}
\frametitle{مقدمه‌ای بر کراس}
\begin{itemize}
\item
کراس یک قالب کاری یادگیری عمیق سطح بالا برای پایتون بوده که شیوه راحتی برای تعریف و آموزش تقریبا هر نوع مدل یادگیری عمیق فراهم می‌کند.
\vspace{0.5cm}
\item
\lr{TensorFlow, Theano} و \lr{CNTK} 
سه پلتفورم اصلی یادگیری عمیق در حال حاضر هستند که هر کد نوشته شده توسط کراس را بدون هیچ تغییری در کد، می‌توان بر روی این موتورها اجرا کرد.
\item
\vspace{0.5cm}
کراس از پایتون‌های ورژن $2.7$ تا $3.6$ پشتیبانی می‌کند.
\vspace{0.5cm}
\item
کراس تحت لیسانس \lr{MIT} توزیع شده که امکان استفاده رایگان در پروژه‌های تجاری را می‌دهد.
\end{itemize}
\end{frame}

\begin{frame}
\frametitle{مقدمه‌ای بر کراس}
کراس ویژگی‌های کلیدی زیر را دارد:
\vspace{0.5cm}
\begin{enumerate}
\item
یک \lr{API} کاربردوست می‌باشد که نمونه‌سازی سریع\LTRfootnote{fast prototyping} مدل‌های یادگیری عمیق را ممکن می‌سازد.
\vspace{0.2cm}
\item
از شبکه‌های کانولوشنی\LTRfootnote{convolutional}، شبکه‌های بازگشتی\LTRfootnote{recurrent} و هر ترکیبی از این دو پشتیبانی می‌کند.
\vspace{0.2cm}
\item
اجازه می‌دهد که یک کد به صورت یکپارچه بر روی \lr{CPU} و \lr{GPU} اجرا شود.
\end{enumerate}
\end{frame}

\end{document}
